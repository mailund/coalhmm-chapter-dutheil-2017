%%%%%%%%%%%%%%%%%%%% author.tex %%%%%%%%%%%%%%%%%%%%%%%%%%%%%%%%%%%
%
% sample root file for your "contribution" to a contributed volume
%
% Use this file as a template for your own input.
%
%%%%%%%%%%%%%%%% Springer %%%%%%%%%%%%%%%%%%%%%%%%%%%%%%%%%%


% RECOMMENDED %%%%%%%%%%%%%%%%%%%%%%%%%%%%%%%%%%%%%%%%%%%%%%%%%%%
\documentclass[graybox]{svmult}

% choose options for [] as required from the list
% in the Reference Guide

\usepackage{mathptmx}       % selects Times Roman as basic font
\usepackage{helvet}         % selects Helvetica as sans-serif font
\usepackage{courier}        % selects Courier as typewriter font
\usepackage{type1cm}        % activate if the above 3 fonts are
                            % not available on your system
%
\usepackage{makeidx}         % allows index generation
\usepackage{graphicx}        % standard LaTeX graphics tool
                             % when including figure files
\usepackage{multicol}        % used for the two-column index
\usepackage[bottom]{footmisc}% places footnotes at page bottom

\usepackage[sort&compress,numbers]{natbib}
\usepackage{todonotes}

%% macros for the manuscript.... %%%%%%%%%%%%%%%%%%%%%%%%%%%%%%%%%%%%%%%%%%%

\renewcommand{\lhd}{\ensuremath{\mathcal{L}}}
\newcommand{\intd}{\ensuremath{\mathrm{\;d\,}}}


%%%%%%%%%%%%%%%%%%%%%%%%%%%%%%%%%%%%%%%%%%%%%%%%%%%%%%%%%%%%%%%%%%%%%%%%%


% see the list of further useful packages
% in the Reference Guide

\makeindex             % used for the subject index
                       % please use the style svind.ist with
                       % your makeindex program

%%%%%%%%%%%%%%%%%%%%%%%%%%%%%%%%%%%%%%%%%%%%%%%%%%%%%%%%%%%%%%%%%%%%%%%%%%%%%%%%%%%%%%%%%

\begin{document}

\title*{Ancestral population genomics with Jocx, a coalescent hidden Markov model}
\author{Jade Yu Cheng and Thomas Mailund}
\institute{Jade Yu Cheng \at FIXME
	\email{name@email.address}
\and Thomas Mailund \at Bioinformatics Research Centre, Aarhus University \email{mailund@birc.au.dk}}
\maketitle

\abstract{FIXME}


\section{Introduction}

Understanding how species form and diverge is a central topic of biology and by observing emerging species today we can understand many of the genetic and environmental processes involved. Through such observations we can understand the underlying forces that drive speciation, but in order to understand how specific speciations occurred in the past, and understand the specifics of how existing species formed, we must make inference from the signals these events have left behind. The study of fossils is a powerful approach here, but not the only avenue to study past speciations; the speciation processes leave fossils in the genome of the resulting species, and through what you might call genetic archaeology we can study past events from the signals they left behind.


The main objectives of the methods we describe in this chapter is to infer demographic parameters, $\Theta$, given genetic data, $D$: $\lhd(\Theta\,|\,D)=\Pr(D\,|\,\Theta)$. Here, we assume that $\Theta$ contains information such as effective population sizes, time points where population structure changes (populations split or admix), or migration rates between populations. We can connect data and demographics through coalescence theory \cite{Hein:2004ta}. This theory gives us a way to assign probability densities to genealogies, densities that depend on the demographic parameters, $f(G\,|\,\Theta)$. Then, if we know the underlying genealogy, we can assign probabilities to observed data using standard algorithms such as \citet{Felsenstein_1981} and get $\Pr(D\,|\,G,\Theta)$. Theoretically, we now simply need to integrate away the nusiance parameter $G$ to get the desired likelihood
\begin{equation}
	\label{eq:likelihood}
	\lhd(\Theta\,|\,D) = \Pr(D\,|\,\Theta) 
	= \int \Pr(D\,|\,G,\Theta) f(G\,|\,\Theta) \intd G .
\end{equation}

In practise, however, the space of all possible genealogies prevents this beyond a small sample size of sequences and for any sizeable length of genetic material, and approximations are needed. The sequential Markov coalescent and coalescent hidden Markov models approximate the likelihood in two steps: assume that sites are independent given the genealogy, i.e.
\begin{equation}
  \Pr(D\,|\,G,\Theta) \approx
  \prod_{i=1}^L \Pr(D_i\,|\,G_i,\Theta)
\end{equation}
where $L$ is the length of the sequence and $D_i$ is the data and $G_i$ the genealogy at site $i$, and assume that the dependency between genealogies is Markov:
\begin{equation}
  \label{eq:markov-genealogy}
  f(G\,|\,\Theta) \approx
  f(G_1\,|\,\Theta)\prod_{i=2}^{L}f(G_{i}\,|\,G_{i-1},\Theta)
  .
\end{equation}

Both assumptions are known to be invalid, but simulation studies indicate that this model captures most important summary statistics from the coalescent \cite{McVean:2005hoa,Marjoram:2006hpa} and that it can be used to accurately infer parameters in various demographic models \cite{Mailund:2011dva,Mailund:2012ewa,Cheng:2015kia}. Because of the form the likelihood now gets,
\begin{equation}
  \label{eq:coalhmm-joint-probability}
  f(D,G\,|\,\Theta) = 
  	f(G_1\,|\,\Theta)
  	\prod_{i=2}^{L}f(G_{i}\,|\,G_{i-1},\Theta)
  	\prod_{i=1}^L \Pr(D_i\,|\,G_i,\Theta)
  	,
\end{equation}
which is the form of a \emph{hidden Markov model}, we can compute the likelihood efficiently using the so-called \emph{Forward} algorithm \cite[chapter 3]{durbin1998biological}.


This efficiency has permitted us and others to apply this approximation to the coalescence to infer demographic parameters on whole genome data~\cite{Li:2011eza,Locke:2011gna, Hobolth:2011dia, Scally:2012ika, Prufer:2012ea, Miller:2012cxa, Abascal:2016cy, PradoMartinez:2013dna, Jonsson:2014fga} in addition to inferring recombination patters \cite{Munch:2014cba,Munch:2014cwa} and scanning for signs of selection \cite{Dutheil:2015kl,Munch:2016dn}.


\todo[inline]{This is probably where we should describe that we only look at pairs and then use a composite likelihood}


\todo[inline]{Update below}

In this chapter we will present the theory underlying our approach to constructing coalescent hidden Markov models and present our current implementations of various models and how you can apply these to your own genomic analyses.



\section{Using Jocx, a CoalHMM inference tool}
\todo[inline]{Tutorial in how to use the software}

\section{CoalHMM results (better title)}
\todo[inline]{Some simulation results}


\bibliographystyle{spbasic}
\bibliography{references.bib}

\end{document}
