\section{Software Usage}

\subsection{Overview}

Jocx is a parameterized model inference framework based on hidden Markov models.
It's constructed on the foundation of coalescence theory with the key approx-
imation that the distribution of local genealogies is Markovian along the
sequence alignment. It combines CoalHMM modeling and several black-box style
optimization subroutines with a special focus on heuristic-based evolution
optimizers. The software package is available on GitHub at the following URL.

{\small{}\begin{verbatim}
  https://github.com/jade-cheng/Jocx.git
\end{verbatim}}

Jocx executes CoalHMM by specifying a model and an optimizer. It uses sequence
alignments in the format of ziphmm directories, which is also prepared by Jocx.
The program prints to standard output the progression of the estimated
parameters and the corresponding log likelihood. The source package contains a
set of Python files, and it requires no installation.

\subsection{Input Data}

Jocs takes pairs of aligned sequences as input. The number of sequence pairs
depends on the CoalHMM model specified for a particular execution. We will
discuss CoalHMM model specification later in this document. For example, for
inference in a two-population isolation scenario, we need a minimal of one pair
of aligned sequences, one from each of the two population. The sequences form an
alignment so they need to match in length and names.

{\small{}\begin{verbatim}
    $ ls
    a.fasta  b.fasta

    $ cat a.fasta | wc -c
    1827

    $ cat b.fasta | wc -c
    1827

    $ head *.fasta -n 7
    ==> a.fasta <==
    >1
    aaaaaaaaaaaaaaaaaaaaaaaaaaaaaaaaaAaaaaaaaaaaaaa
    aaaaaaaaaaaaaaaaaaaaTTaaaaaaaaaaaaaaaaaaaaaaaaa

    >2
    aaaTaaaaaaaaaaaaaaaaaaaaaaaaAaCaaaaaaaaaaaaaaaa
    aaaaaaaaaaaaaaaaaaAaaaaaaaaaaaaaaaaaaaaaaaaaaaa
    ==> b.fasta <==
    >1
    aAaaaaaaaaaaaaaaaaaaaaaaaaaaaaaaaaaaaaaaaaaaaaa
    aaaaaaaaaaaaaaaaaaaaaGaaaaaaGaaaaaaaaaaaaaaaaaa

    >2
    aaaaaaaaaaCaaaaaaaaaaaaaaaaaaaaaaaaaaaaaaaaaaaa
    aaaaaaaaaaaaaaaaaaaaaaaaaaaaaaaTaaaaaaaaaTaaaaaa
\end{verbatim}}

\subsection{ZipHMM Preparation}

We use the ZipHMM algorithm [ref(admix-coalhmm45)] to calculate forward
likelihoods. ZipHMM is shown in previous experiments to give us a speedup in
computing the likelihood of one or two orders of magnitude when analyzing full
genome alignments. To proceed to CoalHMM analysis, user need to prepare the
ZipHMM data directories given pairwise alignments.

This preparation step is customized for each CoalHMM model. In the
aforementioned two-population isolation scenario, we need two aligned sequences
in Fasta format. Executing the following command given the two sequences results
in a ZipHMM directory.

{\small{}\begin{verbatim}
  $ ls
  a.fasta  b.fasta

  $ Jocx.py init . iso a.fasta b.fasta
  # Creating directory: ./ziphmm_iso_a_b
  # creating uncompressed sequence file
  # using output directory "./ziphmm_iso_a_b"
  # parsing "a.fasta"
  # parsing "b.fasta"
  # comparing sequence "1"
  # sequence length: 900
  # creating "./ziphmm_iso_a_b/1.ziphmm"
  # comparing sequence "2"
  # sequence length: 900
  # creating "./ziphmm_iso_a_b/2.ziphmm"
  # Creating 5-state alignment in directory: ./ziphmm_iso_a_b/1.ziphmm
  # Creating 5-state alignment in directory: ./ziphmm_iso_a_b/2.ziphmm
\end{verbatim}}

The first command-line argument, ‘init‘, initializes alignments and zips
sequences that are needed to run an experiment. This command must be executed
once before using the ‘run‘ command. The second argument specified the location
of the output ZipHMM directory.

{\small{}\begin{verbatim}
  $ ls
  a.fasta  b.fasta  ziphmm_iso_a_b

  $ find ziphmm_iso_a_b/
  ziphmm_iso_a_b/
  ziphmm_iso_a_b/1.ziphmm
  ziphmm_iso_a_b/1.ziphmm/data_structure
  ziphmm_iso_a_b/1.ziphmm/nStates2seq
  ziphmm_iso_a_b/1.ziphmm/nStates2seq/5.seq
  ziphmm_iso_a_b/1.ziphmm/original_sequence
  ziphmm_iso_a_b/2.ziphmm
  ziphmm_iso_a_b/2.ziphmm/nStates2seq
  ziphmm_iso_a_b/2.ziphmm/nStates2seq/5.seq
  ziphmm_iso_a_b/2.ziphmm/data_structure
  ziphmm_iso_a_b/2.ziphmm/original_sequence
\end{verbatim}}

The third argument, ‘iso‘, specifies the CoalHMM model of interest. To see the
list of all support models, we use the ‘–help‘ argument. Here ‘iso‘ represent
the two-population two-sequence isolation scenario, shown below.

{\small{}\begin{verbatim}
    $ Jocx.py --help
    :
    ISOLATION MODEL (iso)
        *
       / \ tau
      A   B

    3 params -> tau, coal_rate, recomb_rate
    2 seqs   -> A, B
    1 group  -> AB
    :
\end{verbatim}}

\subsubsection{Sequence Order}

The two-population isolation demographic model is symmetric, so the order of
input Fasta sequences do not matter. This is not always the case. For example,
in a three-population admix model, shown below, the roles populations take are
different. Population C is admixed, and it's formed from ancestral siblings of
the two source populations, A and B. The order of input Fasta sequences,
therefore, needs to match.

{\small{}\begin{verbatim}
    $ Jocx.py --help
    :
    THREE POP ADMIX 2 3 MODEL (admix23)
                      *
                     / \     greedy1_time_1a
    buddy23_time_1a /\  \
                   /  \_/\   buddy23_time_2a
       admix_prop /  <-|  \  iso_time
                A      C   B

    7 params -> iso_time,        buddy23_time_1a,
                buddy23_time_2a, greedy1_time_1a,
                coal_rate, recomb_rate, admix_prop
    3 seqs   -> A, B, C
    3 groups -> AC, BC, AB
    :
\end{verbatim}}

The ‘init‘ command's execution takes aligned Fasta sequences following the order
specified above, i.e. one sequence from population A, followed by one sequence
from population B, followed by one sequence from population C.

{\small{}\begin{verbatim}
  $ ls
  a1.fasta  b1.fasta  c1.fasta

  $ Jocx.py init . admix23 a1.fasta b1.fasta c1.fasta
  # Creating directory: ./ziphmm_admix23_a_c
  # creating uncompressed sequence file
  # using output directory "./ziphmm_admix23_a_c"
  # parsing "a1.fasta"
  # parsing "c1.fasta"
  # comparing sequence "1"
  # sequence length: 900
  # creating "./ziphmm_admix23_a_c/1.ziphmm"
  # comparing sequence "2"
  # sequence length: 900
  # creating "./ziphmm_admix23_a_c/2.ziphmm"
  # Creating 5-state alignment in directory: ./ziphmm_admix23_a_c/2.ziphmm
  # Creating 5-state alignment in directory: ./ziphmm_admix23_a_c/1.ziphmm
  # Creating directory: ./ziphmm_admix23_b_c
  # creating uncompressed sequence file
  # using output directory "./ziphmm_admix23_b_c"
  # parsing "b1.fasta"
  # parsing "c1.fasta"
  # comparing sequence "1"
  # sequence length: 900
  # creating "./ziphmm_admix23_b_c/1.ziphmm"
  # comparing sequence "2"
  # sequence length: 900
  # creating "./ziphmm_admix23_b_c/2.ziphmm"
  # Creating 5-state alignment in directory: ./ziphmm_admix23_b_c/1.ziphmm
  # Creating 5-state alignment in directory: ./ziphmm_admix23_b_c/2.ziphmm
  # Creating directory: ./ziphmm_admix23_a_b
  # creating uncompressed sequence file
  # using output directory "./ziphmm_admix23_a_b"
  # parsing "a1.fasta"
  # parsing "b1.fasta"
  # comparing sequence "1"
  # sequence length: 900
  # creating "./ziphmm_admix23_a_b/1.ziphmm"
  # comparing sequence "2"
  # sequence length: 900
  # creating "./ziphmm_admix23_a_b/2.ziphmm"
  # Creating 5-state alignment in directory: ./ziphmm_admix23_a_b/2.ziphmm
  # Creating 5-state alignment in directory: ./ziphmm_admix23_a_b/1.ziphmm

  $ ls
  a1.fasta  b1.fasta  c1.fasta
  ziphmm_admix23_a_b  ziphmm_admix23_a_c  ziphmm_admix23_b_c

  $ find ziphmm_admix23_*
  ziphmm_admix23_a_b
  ziphmm_admix23_a_b/2.ziphmm
  ziphmm_admix23_a_b/2.ziphmm/data_structure
  ziphmm_admix23_a_b/2.ziphmm/nStates2seq
  ziphmm_admix23_a_b/2.ziphmm/nStates2seq/5.seq
  ziphmm_admix23_a_b/2.ziphmm/original_sequence
  ziphmm_admix23_a_b/1.ziphmm
  ziphmm_admix23_a_b/1.ziphmm/nStates2seq
  ziphmm_admix23_a_b/1.ziphmm/nStates2seq/5.seq
  ziphmm_admix23_a_b/1.ziphmm/data_structure
  ziphmm_admix23_a_b/1.ziphmm/original_sequence
  ziphmm_admix23_a_c
  ziphmm_admix23_a_c/2.ziphmm
  ziphmm_admix23_a_c/2.ziphmm/data_structure
  ziphmm_admix23_a_c/2.ziphmm/nStates2seq
  ziphmm_admix23_a_c/2.ziphmm/nStates2seq/5.seq
  ziphmm_admix23_a_c/2.ziphmm/original_sequence
  ziphmm_admix23_a_c/1.ziphmm
  ziphmm_admix23_a_c/1.ziphmm/data_structure
  ziphmm_admix23_a_c/1.ziphmm/nStates2seq
  ziphmm_admix23_a_c/1.ziphmm/nStates2seq/5.seq
  ziphmm_admix23_a_c/1.ziphmm/original_sequence
  ziphmm_admix23_b_c
  ziphmm_admix23_b_c/2.ziphmm
  ziphmm_admix23_b_c/2.ziphmm/data_structure
  ziphmm_admix23_b_c/2.ziphmm/nStates2seq
  ziphmm_admix23_b_c/2.ziphmm/nStates2seq/5.seq
  ziphmm_admix23_b_c/2.ziphmm/original_sequence
  ziphmm_admix23_b_c/1.ziphmm
  ziphmm_admix23_b_c/1.ziphmm/nStates2seq
  ziphmm_admix23_b_c/1.ziphmm/nStates2seq/5.seq
  ziphmm_admix23_b_c/1.ziphmm/data_structure
  ziphmm_admix23_b_c/1.ziphmm/original_sequence
\end{verbatim}}

\subsubsection{Sequences per Population} In the two examples above, each
population contributes one and only one sequence to the CoalHMM model's
construction. Jocx also has models that support two sequences per population.
This is to better take advantage of the genomic data that's available to the
researcher.

{\small{}\begin{verbatim}
  $ Jocx.py --help
    :
    THREE POP ADMIX 2 3 MODEL 6 HMM (admix23-6hmm)
                      *
                     / \     greedy1_time_1a
    buddy23_time_1a /\  \
                   /  \_/\   buddy23_time_2a
       admix_prop /  <-|  \  iso_time
                 A1   C1   B1
                 A2   C2   B2

    7 params -> iso_time,        buddy23_time_1a,
                buddy23_time_2a, greedy1_time_1a,
                coal_rate, recomb_rate, admix_prop
    6 seqs   -> A1, A2, B1, B2, C1, C2
    6 groups -> A1C1, B1C1, A1B1, A1A2, B1B2, C1C2
    :
\end{verbatim}}

In this example, we demonstrate the same admixture demographic model but with
each population contributing two sequences to form six pairwise alignments,
which are then used to construct six HMMs for the inference.

{\small{}\begin{verbatim}
  $ ls
  a1.fasta  a2.fasta  b1.fasta  b2.fasta  c1.fasta  c2.fasta

  $ Jocx.py init . admix23-6hmm a1.fasta a2.fasta b1.fasta b2.fasta c1.fasta c2.fasta
  # Creating directory: ./ziphmm_admix23-6hmm_a1_c1
  # creating uncompressed sequence file
  # using output directory "./ziphmm_admix23-6hmm_a1_c1"
  # parsing "a1.fasta"
  # parsing "c1.fasta"
  # comparing sequence "1"
  # sequence length: 900
  # creating "./ziphmm_admix23-6hmm_a1_c1/1.ziphmm"
  # comparing sequence "2"
  # sequence length: 900
  # creating "./ziphmm_admix23-6hmm_a1_c1/2.ziphmm"
  # Creating 5-state alignment in directory: ./ziphmm_admix23-6hmm_a1_c1/2.ziphmm
  # Creating 5-state alignment in directory: ./ziphmm_admix23-6hmm_a1_c1/1.ziphmm
  # Creating directory: ./ziphmm_admix23-6hmm_b1_c1
  # creating uncompressed sequence file
  # using output directory "./ziphmm_admix23-6hmm_b1_c1"
  # parsing "b1.fasta"
  # parsing "c1.fasta"
  # comparing sequence "1"
  # sequence length: 900
  # creating "./ziphmm_admix23-6hmm_b1_c1/1.ziphmm"
  # comparing sequence "2"
  # sequence length: 900
  # creating "./ziphmm_admix23-6hmm_b1_c1/2.ziphmm"
  # Creating 5-state alignment in directory: ./ziphmm_admix23-6hmm_b1_c1/1.ziphmm
  # Creating 5-state alignment in directory: ./ziphmm_admix23-6hmm_b1_c1/2.ziphmm
  # Creating directory: ./ziphmm_admix23-6hmm_a1_b1
  # creating uncompressed sequence file
  # using output directory "./ziphmm_admix23-6hmm_a1_b1"
  # parsing "a1.fasta"
  # parsing "b1.fasta"
  # comparing sequence "1"
  # sequence length: 900
  # creating "./ziphmm_admix23-6hmm_a1_b1/1.ziphmm"
  # comparing sequence "2"
  # sequence length: 900
  # creating "./ziphmm_admix23-6hmm_a1_b1/2.ziphmm"
  # Creating 5-state alignment in directory: ./ziphmm_admix23-6hmm_a1_b1/2.ziphmm
  # Creating 5-state alignment in directory: ./ziphmm_admix23-6hmm_a1_b1/1.ziphmm
  # Creating directory: ./ziphmm_admix23-6hmm_a1_a2
  # creating uncompressed sequence file
  # using output directory "./ziphmm_admix23-6hmm_a1_a2"
  # parsing "a1.fasta"
  # parsing "a2.fasta"
  # comparing sequence "1"
  # sequence length: 900
  # creating "./ziphmm_admix23-6hmm_a1_a2/1.ziphmm"
  # comparing sequence "2"
  # sequence length: 900
  # creating "./ziphmm_admix23-6hmm_a1_a2/2.ziphmm"
  # Creating 5-state alignment in directory: ./ziphmm_admix23-6hmm_a1_a2/1.ziphmm
  # Creating 5-state alignment in directory: ./ziphmm_admix23-6hmm_a1_a2/2.ziphmm
  # Creating directory: ./ziphmm_admix23-6hmm_b1_b2
  # creating uncompressed sequence file
  # using output directory "./ziphmm_admix23-6hmm_b1_b2"
  # parsing "b1.fasta"
  # parsing "b2.fasta"
  # comparing sequence "1"
  # sequence length: 900
  # creating "./ziphmm_admix23-6hmm_b1_b2/1.ziphmm"
  # comparing sequence "2"
  # sequence length: 900
  # creating "./ziphmm_admix23-6hmm_b1_b2/2.ziphmm"
  # Creating 5-state alignment in directory: ./ziphmm_admix23-6hmm_b1_b2/1.ziphmm
  # Creating 5-state alignment in directory: ./ziphmm_admix23-6hmm_b1_b2/2.ziphmm
  # Creating directory: ./ziphmm_admix23-6hmm_c1_c2
  # creating uncompressed sequence file
  # using output directory "./ziphmm_admix23-6hmm_c1_c2"
  # parsing "c1.fasta"
  # parsing "c2.fasta"
  # comparing sequence "1"
  # sequence length: 900
  # creating "./ziphmm_admix23-6hmm_c1_c2/1.ziphmm"
  # comparing sequence "2"
  # sequence length: 900
  # creating "./ziphmm_admix23-6hmm_c1_c2/2.ziphmm"
  # Creating 5-state alignment in directory: ./ziphmm_admix23-6hmm_c1_c2/1.ziphmm
  # Creating 5-state alignment in directory: ./ziphmm_admix23-6hmm_c1_c2/2.ziphmm

  $ ls
  a1.fasta  b1.fasta  c1.fasta
  a2.fasta  b2.fasta  c2.fasta
  ziphmm_admix23-6hmm_a1_a2  ziphmm_admix23-6hmm_a1_c1  ziphmm_admix23-6hmm_b1_c1
  ziphmm_admix23-6hmm_a1_b1  ziphmm_admix23-6hmm_b1_b2  ziphmm_admix23-6hmm_c1_c2
\end{verbatim}}

\subsection{MLE Optimization}

Jocx implements three optimization subroutines, Nelder-Mead (NM), Genetic
Algorithm (GA), and Particle Swarm Optimization (PSO). After preparing the
ZipHMM directories, user can choose to proceed to CoalHMM parametrized model
inference using one of these three black-box optimizers.

We use the two-population isolation CoalHMM as an example in this sections. The
first command-line argument, ‘run‘, runs an experiment. Before running an
experiment, the appropriate directories must be initialized using the ‘init‘
command. The second argument specifies the directory where the ZipHMM data
directories can be found. The third argument, ‘iso‘, specified the CoalHMM model
of interest. The ZipHMM directories found in the given directory need to match
with the model specification. The fourth argument, ‘nm‘, ‘ga‘, or ‘pso‘,
specifies the optimization method used for the execution.

The rest of the arguments are used to determine the initial parameter condition.
In NM they are the initial simplex vertices. In PSO and GA, the first generation
of solutions are sampled from the range +/- 100x of the given values. The length
of these parameters matches the specified CoalHMM model.

{\small{}\begin{verbatim}
    $ Jocx.py --help
    :
    3 params -> tau, coal_rate, recomb_rate
    :
\end{verbatim}}

In the two-population isolation model, represented by the ‘iso‘ argument, we
infer three model parameters, the population split time, the coalescent rate,
and the recombination rate. Populations are modeled to have the same coalescent
rate. The order of these three parameters follows the ASCII diagram printed
using the ‘– help‘ argument, shown above. For this model we first have the split
time, then the coalescent rate, and finally the recombination rate.

\subsubsection{NM}

NM was introduced by John Nelder and Roger Mead in 1965 [ref(thesis25)] as a
technique to minimize an objective function in a many-dimensional space. This
method uses several algorithm coefficients to determine the amount of effect of
possible actions. They are the reflection coefficient r, the expansion
coefficient c, the contraction coefficient g, and the shrinkage coefficient s.
Standard values recommended in [ref(thesis3)] are r = 1, c = 2, g = 1/2, and s =
1/2.

{\small{}\begin{verbatim}
    $ Jocx.py run . iso nm 0.0001 1000 0.1
    # algorithm            = _NMOptimiser
    # timeout              = None
    # max_executions       = 1
    #
    # 2017-10-11 11:29:08.069462
    :
    # execution state score param0 param1 param2
    0 init -38.2023478685 0.000376954454165 7480.36836670 0.337649514816
    1 fmin-in -40.5337262711 0.000385595244114 661.208520686 0.920281817958
    1 fmin-cb -40.3804021200 0.000385595244114 694.268946721 0.920281817958
    :
    1 fmin-cb -37.8927822292 0.000695082517418 200504630.601 32081.6528250
    Optimization terminated successfully.
         Current function value: 37.892782
         Iterations: 262
         Function evaluations: 533
    1 fmin-out -37.8927822292 0.000695082517418 200504630.601 32081.652825
\end{verbatim}}

In the output of NM's execution, we have a final report of whether or not the
execution was successful together with the optimal solution. Different reasons
contribute to the failure of an execution. The number of parameters to solve is
a major reason. When CoalHMM models become complex, the number of parameters
increase to a level beyond NM's capability. Initial parameters is another major
reason for failed NM executions.

\subsubsection{GA}

GA was introduced by John Holland first introduced in the 1970s [ref(thesis13)].
The idea is to encode each solution as a chromosome-like data structure and
operate on them through actions analogous to genetic alterations, which usually
involves selection, recombination, and mutation. For each type of alteration,
people have developed different techniques.

{\small{}\begin{verbatim}
  $ Jocx.py run . iso ga 0.0001 1000 0.1
  # algorithm            = _GAOptimiser
  # timeout              = None
  # elite_count          = 1
  # population_size      = 50
  # initialization       = UniformInitialisation
  # selection            = TournamentSelection
  # tournament_ratio     = 0.1
  # selection_ratio      = 0.75
  # mutation             = GaussianMutation
  # point_mutation_ratio = 0.15
  # mu                   = 0.0
  # sigma                = 0.01
  #
  # 2017-10-23 10:31:32.821761
  #
  # param0 = (1.0000000000000016e-05, 0.001)
  # param1 = (99.99999999999996, 10000.0)
  # param2 = (0.009999999999999995, 1.0)
  #
  #
  # POPULATION FOR GENERATION 1
  # average_fitness = -5.32373335161
  # min_fitness     = -10.7962322739
  # max_fitness     = -0.613544122419
  #
  # gen idv       fitness      param0         param1      param2
      1   1   -0.61354412  0.00002825  6305.95175380  0.04139445
      1   2   -1.38710619  0.00004282  2182.61708962  0.03027973
      1   3   -4.45085424  0.00001133   254.73764392  0.01081756
      1   4   -9.37092993  0.00067074   116.84983427  0.13757425
      1   5  -10.79623227  0.00071728   142.34535478  0.81564586
      :
  #
  # POPULATION FOR GENERATION 2
  # average_fitness = -5.83495296756
  # min_fitness     = -10.5697879572
  # max_fitness     = -0.613544122419
  #
  # gen idv       fitness      param0         param1      param2
      2   1   -0.61354412  0.00002825  6305.95175380  0.04139445
      2   2   -0.61382451  0.00002825  6305.95175380  0.13757425
      2   3   -6.89850999  0.00002825   116.84983427  0.14110664
      2   4  -10.47909826  0.00067074   145.01523656  0.81564586
      2   5  -10.56978796  0.00067074   142.34535478  0.81564586
      :
  :
\end{verbatim}}

In the output of GA's execution, we have multiple generations of solutions, and
multiple solutions per generation. Solutions in each generation is ordered by
the fitness, i.e. best solution is at the top. The final solution is, therefore,
the first solution in the last generation.

\subsubsection{PSO}

PSO was introduced by Eberhart and Kennedy in 1995 [ref(thesis6)] as an
optimization technique relying on stochastic processes, similar to GA. As its
name implies, each individual solution mimics a particle in a swarm. Each
particle holds a velocity and keeps track of the best positions it has
experienced and best position the warm has experienced. The former encapsulates
the social influence, i.e. a force pulling towards the swarm's best. The latter
encapsulates the cognitive influence, i.e. a force pulling towards the
particle's best. Both forces act on the velocity and drive the particle through
a hyper parameter space.

{\small{}\begin{verbatim}
  $ Jocx.py run . iso pso 0.0001 1000 0.1
  # algorithm            = _PSOptimiser
  # timeout              = None
  # max_iterations       = 50
  # particle_count       = 50
  # max_initial_velocity = 0.02
  # omega                = 0.9
  # phi_particle         = 0.3
  # phi_swarm            = 0.1
  #
  # 2017-10-23 10:32:29.123305
  #
  # param0 = (1.0000000000000016e-05, 0.001)
  # param1 = (99.99999999999996, 10000.0)
  # param2 = (0.009999999999999995, 1.0)
  #
  #
  # PARTICLES FOR ITERATION 1
  # swarm_fitness           = -0.832535308472
  # best_average_fitness    = -4.40169918533
  # best_minimum_fitness    = -9.77654933959
  # best_maximum_fitness    = -0.832535308472
  # current_average_fitness = -4.40169918533
  # current_minimum_fitness = -9.77654933959
  # current_maximum_fitness = -0.832535308472
  #
  #                                           best-   best-     best-     best-
  # gen idv  fitness  param0   param1  param2 fitness param0    param1    param2
      1   0  -0.83  0.000044  4619.31  0.20   -0.83   0.000044  4619.31   0.20
      1   1  -0.86  0.000048  4502.80  0.26   -0.86   0.000048  4502.80   0.26
      1   2  -0.89  0.000061  4669.48  0.58   -0.89   0.000061  4669.48   0.58
      1   3  -1.10  0.000035  2970.77  0.31   -1.10   0.000035  2970.77   0.31
      1   4  -1.46  0.000057  2148.93  0.15   -1.46   0.000057  2148.93   0.15
      :
  #
  # PARTICLES FOR ITERATION 2
  # swarm_fitness           = -0.810479293858
  # best_average_fitness    = -4.02436023707
  # best_minimum_fitness    = -9.12434788412
  # best_maximum_fitness    = -0.810479293858
  # current_average_fitness = -4.02984771812
  # current_minimum_fitness = -9.12434788412
  # current_maximum_fitness = -0.810479293858
  #
  #                                           best-   best-     best-     best-
  # gen idv  fitness  param0   param1  param2 fitness param0    param1    param2
      2   0  -0.81  0.000045  4854.87  0.25   -0.81   0.000045  4854.87   0.25
      2   1  -0.82  0.000040  4622.38  0.21   -0.82   0.000040  4622.38   0.21
      2   2  -0.91  0.000064  4599.97  0.59   -0.89   0.000061  4669.48   0.58
      2   3  -1.12  0.000038  2917.40  0.29   -1.10   0.000035  2970.77   0.31
      2   4  -1.39  0.000058  2308.29  0.14   -1.39   0.000058  2308.29   0.14
      :
  :
\end{verbatim}}

In the output of the PSO's execution, we have multiple generations and multiple
particles (solutions) per generation. Each particle contains two sets of
solutions, the current solution and the best solution that this particle has
encountered throughout the PSO's execution. The latter is always better than the
former. Similar to GA, each generation is ordered by the particles' fitness. The
final solution is, therefore, the second solution of the first particle in the
last generation.
